\documentclass[11pt,a4paper,oneside]{article}
\usepackage[utf8]{inputenc}
\usepackage{t1enc} % hyphenate accented chars
\usepackage[hungarian]{babel}
\usepackage{../fedlap}
\usepackage{fancyhdr} % elofej, elolab
\usepackage{graphicx}
\setcounter{secnumdepth}{3} % enable subsubsection

% hasonlitson a doc verziora
\addtolength{\voffset}{-1cm}

% cim
\csapat{nand}{404}
\konzulens{Bozóki Szilárd}
\datum{\today}

% csapattagok
\taga{Berki Endre}{HQNHER}{berkiendre@gmail.com}
\tagb{Fodor Bertalan Ferenc}{H4T1UX}{foberci@gmail.com}
\tagc{Kádár András}{JFENWR}{arycika@gmail.com}
\tagd{Thaler Benedek}{EDDO10}{thalerbenedek@gmail.com}

\setlength{\headheight}{1.3em}
\setlength{\headsep}{2em}

% elofej, elolab
\fancyhf{}

\fancyhead[OL] { \tiny \leftmark{} }
\fancyhead[OR] { \tmpcsapat }

\fancyfoot[OR] { \thepage }
\fancyfoot[OL] { \tmpdatum }

\pagestyle{fancy}


\begin{document}

\anyag{2. Követelmény, projekt, funkcionalitás}
\fedlap
\tableofcontents

\addtocounter{section}{1}
\section{Követelmény, projekt, funkcionalitás}

\subsection{Követelmény definíció}

    \subsubsection{A program bemutatása, alapvető feladatai}
Az elkészített program egy nagymértékben logikai, kismértékben ügyességi játék, mely koncepcióját tekintve teljesen megfelel a Ragtime Games által fejlesztett Continuity játéknak\footnote{http://continuitygame.com/playcontinuity.html}. A játék során a játékos több pályán keresztül tesztelheti logikai és problémamegoldó készségét. Az előre kialakított pályákon az előrehaladás a keretek egymáshoz képest történő elmozdításával, a kereteken belül pedig egy figurával 2 dimenzióban végzett mozgással történik. Egy pálya akkor tekinthető teljesítettnek, ha a felhasználó összegyűjtötte a pályán található összes kulcsot és ezek felhasználásával bemegy a pálya végét szimbolizáló ajtón.

Minden pálya egy n*n-es táblából áll, amin n*n-1 keret található. Az összeillő keretek között átjárás van, az üres hellyel szomszédos keretek átmozgathatóak az üres helyre. A játékos által irányított figurának a keretekben található kulcsokat kell összeszednie, hogy azzal a szintén a keretekben található ajtót kinyitva a pályát teljesítse.

    \subsubsection{Felhasználói felület}
A kész program grafikus felhasználói felülettel rendelkezik, melyen az indítással és működéssel kapcsolatos funkciók egy menüsoron keresztül érhetőek el, egér vagy billentyűzet segítségével. Maga a játék kizárólag billentyűzettel irányítható.

    \subsubsection{A program futtatásának követelményei}
A program futtatásához szükséges a futtató számítógépre telepített Java Runtime Environment (JRE), mely ingyenesen beszerezhető a gyártó honlapjáról\footnote{http://java.com/en/download/index.jsp} a legtöbb platformra. A program hardware követelménye megegyezik az Oracle által meghatározott minimum-konfigurációval\footnote{http://www.java.com/en/download/help/sysreq.xml}, mely szükségessé tesz egy kompatibilis operációs rendszert (Windows, Linux), rendszertől függően 64-128 MB memóriát és hozzávetőlegesen 100 MB szabad helyet.

    \subsubsection{A program telepítése}
A program futtatásához a kész terméket csupán a futtató számítógép merevlemezére kell másolni, további telepítés vagy beállítás nem szükséges, amennyiben a futtató számítógép megfelel a futtatás követelményeinek.

    \subsubsection{A program fordítása}
A program fordításához szükség van a \texttt{javac} programra, mellyel a \texttt{*.java} fájlok lefordíthatóak, és a futtatható \texttt{jar} fájl elkészíthető.
    
\subsection{Projekt terv}

    \subsubsection{A felhasznált fejlesztőeszközök}
	A csapat választása az Eclipse\footnote{http://eclipse.org/} integrált fejlesztői rendszer használatára esett, elsosorban annak testreszabhatósága, univerzalitása miatt. Felhasználásra került továbbá a Visual Paradigm Community Edition\footnote{http://www.visual-paradigm.com/} kiadása, mely képes UML-diagramokból Java-kódot készíteni, így hatékonyan támogatja a fejlesztőcsapat munkáját, és illeszkedik annak filozófiájához, miszerint elsősorban a kiadott feladat részletes analízisén és tervezésen van a hangsúly, a kész programkód pedig ezzel kell, hogy szoros kapcsolatban legyen.
A dokumentáció LATEX leírónyelv használatával nyerte el végleges formáját. A projekt-menedzsmentet és a verziókezelést egy github\footnote{https://github.com/} névre hallgató felületen keresztül oldjuk meg.

    \subsubsection{A fejlesztőcsapat tagjai, feladatai}
	\begin{center}
	\begin{tabular} {| l | c | }
		\hline
		Név & Feladatok\\
		\hline
		Thaler Benedek & csapatvezetés, dokumentáció, diagramok készítése, kódírás \\ 
		\hline
		Berki Endre & dokumentáció, diagramok készítése, kódírás \\
		\hline
		Fodor Bertalan & dokumentáció, diagramok készítése, kódírás \\
		\hline
		Kádár András & dokumentáció, diagramok készítése, kódírás \\
		\hline
	\end{tabular}
	\end{center}

    \subsubsection{Kommunikációs modell}

    \subsubsection{Fejlesztési mérföldkövek, ütemterv}

	\begin{description}
		\item[Skeleton] \hfill \\
			Az első fontosabb mérföldkő, a szó jelentése: váz, vázrendszer. A program vázának megalkotása. A program modelljének megtervezése, annak helyességének és részletességének ellenőrzése, maga a tervezési folyamat során a gondos és odafigyelő döntéshozás, mert ezek a már elkészült program minőségét nagyban befolyásolják. Ez egy időigényes folyamat, azonban amennyiben sikeresen zárul, a projekt további folyamatai során esetlegesen felmerülő komplikációk száma minimalizálható.
		\item[Prototípus] (a továbbiakban proto)\hfill \\
			A második mérföldkő a program működőképes változata, ami még a grafikát nem tartalmazza. A mérföldkő elérése a program egészének tesztelését teszi lehetővé. A fejlesztés ezen része lezárultnak tekinthető, az algoritmusok végleges formájukba kerülnek, így egy a grafikus felülettül elkülönülő kész programot jelent.
		\item[Grafikus felület] (a továbbiakban GUI)\hfill \\
			A harmadik és egyben utolsó mérföldkövet a grafikus felület befejezése jelenti. Figyelni kell arra, hogy a már kész, jól működő modell köré egy olyan felhasználói környezetet teremtsünk, ami a program használatának hatékonyságát minél jobban maximalizálja. A GUI kialakításánál szem előtt kell tartani a használhatóságot, ergonómiát és a könnyen tanulhatóságot.
	\end{description}

    \subsubsection{Határidők}
	\begin{center}
	\begin{tabular*}{0.65\textwidth}{@{\extracolsep{\fill}} | l | l | }
		\hline
		febr. 10. & 14 h - csapatok regisztrációja\\
		\hline
		febr. 20. & Követelmény, projekt, funkcionalitás - beadás\\
		\hline
		febr. 27. & Analízis modell kidolgozása 1. - beadás\\
		\hline
		márc. 5. & Analízis modell kidolgozása 2. - beadás\\
		\hline
		márc. 12. & Szkeleton tervezése - beadás\\
		\hline
		márc. 19. & Szkeleton - beadás\\
		\hline
		márc. 26. & Prototípus koncepciója - beadás\\
		\hline
		ápr. 2. & Részletes tervek - beadás\\
		\hline
		ápr. 16. & Prototípus - beadás\\
		\hline
		ápr. 23. & Grafikus felület specifikációja - beadás\\
		\hline
		máj. 7. & Grafikus változat - beadás\\
		\hline
		máj. 11. & Összefoglalás - beadás\\
		\hline
	\end{tabular*}
	\end{center}

    \subsubsection{Átadás}
	A dokumentációt nyomtatott formában minden héten a konzulensnek le kell adni a konzultáció időpontjában. Ezen kívül a három mérföldkő alkalmával a programkódok is bemutatásra kerülnek az ezen célra kijelölt laboratóriumok gépeinél. A végső átadás a félév végeztével az aktualizált dokumentációval és forráskódokkal feltöltésre kerül a tárgy honlapjára\footnote{http://www.iit.bme.hu/hercules}.

   \subsubsection{Kockázatelemzés}
	\begin{description}
	\item[Valószínűségek osztályozása:] \hfill \\
		\begin{description}
			\item[Alacsony] Közelítőleg 0-20\%-os eséllyel bekövetkező esemény
			\item[Közepes] Közelítőleg 20-50\%-os eséllyel bekövetkező esemény
			\item[Biztos] Közelítőleg 50-100\%-os eséllyel bekövetkező esemény
		\end{description}
	\item[Hatások osztályozása:] \hfill \\
		\begin{description}
			\item[Elhanyagolható] Az esemény bekövetkezése nem okoz különösebb problémát, az esetleges javítása nem igényel komoly munkálatokat.
			\item[Enyhe] Az esemény bekövetkeztével okozott kár legfeljebb néhány munkaórával javítható, visszaállítható.
			\item[Közepes] A csapat több tagjának összehangolt munkáját igénylő probléma, melynek megoldása komolyabb erőfeszítéseket igényel. Ezen változtatások már hatással lehetnek a heti ütemtervre, és a fejlesztőcsapat közös
megbeszélését igényelhetik.
			\item[Komoly] A projekt sikeres kimenetelét vagy a csapat integritását fenyegető esemény, mely alapvető változtatásokkal jár. Az ilyen jellegű probléma a csapat azonnali tanácskozását igényli.
		\end{description}
	\end{description}

	Eseménytáblázat:
	\begin{center}
		\begin{tabular}{| l | l | l |}
			\hline
			\textbf{Esemény}			&	\textbf{Valószínűség}	&	\textbf{Hatás}\\
			\hline
			Specifikációváltozás			&	Biztos				&	Enyhe\\
			\hline
			Javítandó heti beadandó		&	Közepes			&	Enyhe\\
			\hline
			Sikertelen heti beadandó		&	Alacsony			&	Közepes\\
			\hline
			Csapattag kiválás			&	Alacsony			&	Komoly\\
			\hline
			Csapattag betegsége, távolléte	&	Alacsony			&	Közepes\\
			\hline
			Hardvermeghibásodás		&	Alacsony			&	Elhanyagolható\\
			\hline
			Szoftvermeghibásodás		&	Közepes			&	Enyhe\\
			\hline
			Határidorol lecsúszás		&	Alacsony			&	Közepes\\
			\hline
			Tanácskozásról hiányzás		&	Közepes			&	Közepes\\
			\hline
		\end{tabular}
	\end{center}

   \subsubsection{Egyéb fontos megjegyzések}
	A fejlesztés lefordítása és bemutatása a HSZK laborjaiban történik, így a program forráskódjának kompatibilisnek kell lennie a Java Development Kit ennek megfelelő korábbi verziójával. A maximális kompatibilitás elérése érdekében a fejlesztés is pontosan ezeken a verziószámú platformokon történik majd, azaz ???-es Java SDK és ??? verziójú JRE kerül felhasználásra.

    \subsubsection{Szükséges dokumentációk}
	\begin{enumerate}
	\item Követelmény, projekt, funkcionalitás
	\item Analízis modell kidolgozása 1.
	\item Analízis modell kidolgozása 2.
	\item Szkeleton tervezése
	\item Szkeleton
	\item Prototípus koncepciója
	\item Részletes tervek
	\item Prototípus
	\item Grafikus felület specifikációja
	\item Grafikus változat
	\end{enumerate}
 
\subsection{Feladatleírás}
\subsection{Szótár}
\subsection{Essential use-case-ek}
    %insert figure
	%\begin{figure}[h]
	%	\begin{center}
	%		\includegraphics[scale=0.5]{----FILENAME-HERE----}
	%		\caption{UML diagram test}
	%	\end{center}
	%\end{figure}

\end{document}
