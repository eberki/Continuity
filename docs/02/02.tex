\documentclass[11pt,a4paper,oneside]{article}
\usepackage[utf8]{inputenc}
\usepackage{t1enc} % hyphenate accented chars
\usepackage[hungarian]{babel}
\usepackage{../fedlap}
\usepackage{fancyhdr} % elofej, elolab
\usepackage{graphicx}
\setcounter{secnumdepth}{3} % enable subsubsection

% hasonlitson a doc verziora
\addtolength{\voffset}{-1cm}

% cim
\csapat{nand}{404}
\konzulens{Bozóki Szilárd}
\datum{\today}

% csapattagok
\taga{Berki Endre}{HQNHER}{berkiendre@gmail.com}
\tagb{Fodor Bertalan Ferenc}{H4T1UX}{foberci@gmail.com}
\tagc{Kádár András}{JFENWR}{arycika@gmail.com}
\tagd{Thaler Benedek}{EDDO10}{thalerbenedek@gmail.com}

\setlength{\headheight}{1.3em}
\setlength{\headsep}{2em}

% elofej, elolab
\fancyhf{}

\fancyhead[OL] { \tiny \leftmark{} }
\fancyhead[OR] { \tmpcsapat }

\fancyfoot[OR] { \thepage }
\fancyfoot[OL] { \tmpdatum }

\pagestyle{fancy}


\begin{document}

\anyag{2. Követelmény, projekt, funkcionalitás}
\fedlap
\tableofcontents

\addtocounter{section}{1}
\section{Követelmény, projekt, funkcionalitás}

\subsection{Követelmény definíció}

    \subsubsection{A program bemutatása, alapvető feladatai}
Az elkészített program egy nagymértékben logikai, kismértékben ügyességi játék, mely koncepcióját tekintve teljesen megfelel a Ragtime Games által fejlesztett Continuity játéknak\footnote{http://continuitygame.com/playcontinuity.html}. A játék során a játékos több pályán keresztül tesztelheti logikai és problémamegoldó készségét. Az előre kialakított pályákon az előrehaladás a keretek egymáshoz képest történő elmozdításával, a kereteken belül pedig egy figurával 2 dimenzióban végzett mozgással történik. Egy pálya akkor tekinthető teljesítettnek, ha a felhasználó összegyűjtötte a pályán található összes kulcsot és ezek felhasználásával bemegy a pálya végét szimbolizáló ajtón.

Minden pálya egy n*n-es táblából áll, amin n*n-1 keret található. Az összeillő keretek között átjárás van, az üres hellyel szomszédos keretek átmozgathatóak az üres helyre. A játékos által irányított figurának a keretekben található kulcsokat kell összeszednie, hogy azzal a szintén a keretekben található ajtót kinyitva a pályát teljesítse.

    \subsubsection{Felhasználói felület}
A kész program grafikus felhasználói felülettel rendelkezik, melyen az indítással és működéssel kapcsolatos funkciók egy menüsoron keresztül érhetőek el, egér vagy billentyűzet segítségével. Maga a játék kizárólag billentyűzettel irányítható.

    \subsubsection{A program futtatásának követelményei}
A program futtatásához szükséges a futtató számítógépre telepített Java Runtime Environment (JRE), mely ingyenesen beszerezhető a gyártó honlapjáról\footnote{http://java.com/en/download/index.jsp} a legtöbb platformra. A program hardware követelménye megegyezik az Oracle által meghatározott minimum-konfigurációval\footnote{http://www.java.com/en/download/help/sysreq.xml}, mely szükségessé tesz egy kompatibilis operációs rendszert (Windows, Linux), rendszertől függően 64-128 MB memóriát és hozzávetőlegesen 100 MB szabad helyet.

    \subsubsection{A program telepítése}
A program futtatásához a kész terméket csupán a futtató számítógép merevlemezére kell másolni, további telepítés vagy beállítás nem szükséges, amennyiben a futtató számítógép megfelel a futtatás követelményeinek.

    \subsubsection{A program fordítása}
A program fordításához szükség van a \texttt{javac} programra, mellyel a \texttt{*.java} fájlok lefordíthatóak, és a futtatható \texttt{jar} fájl elkészíthető.
    
\subsection{Projekt terv}
\subsection{Feladatleírás}
\subsection{Szótár}

\begin{description}

    \item[Játékos] A programot kezelő felhasználó
    \item[Számítógép] A programot futtató számítógép, mely megfelel a futtatás követelményeinek
    \item[Monitor] A \emph{Számítógéphez} csatlakoztatott képernyő, melyen a futtatott program megjelenik
    \item[Pálya] A játékban való előrehaladás egysége. Minden pálya egy n*k méretű táblázatból áll, melyben n*k-1 előre meghatározott \emph{Keret} előre meghatározott helyen található. A fennmaradó helyen egy \emph{Üres keret} található
    \item[Kezdőpont] A \emph{Pálya} egy pontja, ahol \emph{Stickman} először megjelenik a pálya betöltésekor.
    \item[Keret] A \emph{Pálya} eleme, egyszerre több is megjelenik, \emph{Elem}eket tartalmaz. Egymáshoz képest \emph{Átrendez}hetőek. Két érintkező keret között egyértelműen megállapítható, hogy \emph{Átjárhatóak}-e. Amennyiben \emph{Stickman} a keret alsó szélét érinti úgy, hogy az \emph{Aktuális keret} nem \emph{Átjárható} az alatta található kerettel, vagy nincs alatta keret, akkor \emph{Stickman} \emph{Kiesik}.
    \item[Átjárható] Két \emph{Keret} átjárható, ha az érintkező oldalon a \emph{Platform}ok minden pontban azonos magasságban vannak.
    \item[Aktuális keret] \emph{Keret}, amiben a játékos lába tartózkodik.
    \item[Üres keret] Az üres keret biztosít lehetőséget arra, hogy vele helyetcserélve a \emph{Keret}ek átrendezhetőek legyenek.
    \item[Elem] Egy keret dinamikus vagy statikus alkotórészei; \emph{Stickman, Kulcs, Ajtó, Platform}
    \item[Stickman] A játékban a \emph{Játékos} által irányított emberformájú figura.
    \item[Kulcs] A \emph{Játékos} által \emph{Megszerez}hető \emph{Elem}. A \emph{Pályá}n található összes kulcs megszerzése után van lehetőség az \emph{Ajtó} \emph{Kinyit}ására.
    \item[Ajtó] \emph{Elem}, melyet \emph{Kinyit}va a pálya \emph{Teljesít}ettnek tekinthető.
    \item[Platform] \emph{Elem}, mely meghatározza a \emph{Keret} bejárható és nem bejárható részeit. Egy 2 dimenziós objektum, melyen a \emph{Játékos} irányíthatása szerint \emph{Stickman} mozoghat.
    \item[Megszerez] \emph{Stickman} egy megszerezhető \emph{Elem}et megszerez, ha azt megérinti. Egy elem csak egyszer szerezhető meg, el nem veszíthető.
    \item[Kinyit] \emph{Stickman} az \emph{Ajtó}t kinyithatja, ha \emph{Megszerez}te a \emph{Pályá}n található összes \emph{Kulcs}ot. A kinyitás módja az ajtó megérintése. A kinyitás bekövetkeztekor a \emph{Pálya} \emph{Teljesít}ettnek tekinthető.
    \item[Teljesít (Pályát)] A \emph{Játékos} teljesíti a \emph{Pályát}, ha \emph{Kinyit}ja a pálya \emph{Ajtaját}. Teljesítés után a \emph{Következő pályá}ra léphet.
    \item[Kiesik] Ha \emph{Stickman} kiesik, akkor eltűnik, majd megjelenik az utolsó érintett \emph{Ellenőrzőpont}nál
    \item[Ellenőrzőpont] Ellenőrzőpont a \emph{Pálya} \emph{Kezdőpont}ja, valamint minden \emph{Kulcs} helye.
    \item[Átrendez] Egy \emph{Keret} átrendezhető, ha mellette \emph{Üres keret} található és \emph{Stickman} csupán egy keretben jelenik meg. Ekkor a két \emph{Keret} helye felcserélhető. Az átrendezést a \emph{Játékos} vezérli.
    \item[Kontextus] Meghatározza, hogy a \emph{Játékos} az összes \emph{Keret}et, (\emph{Pálya kontextus}) vagy csak az \emph{Aktuális keret}et (\emph{Keret kontextus}) lássa.
    \item[Pálya kontextus] A \emph{Játékos} az összes \emph{Keret}et látja, és vezérelheti az \emph{Átrendez}ést.
    \item[Keret kontextus] A \emph{Játékos} csak az \emph{Aktuális keret}et látja, vezérelheti \emph{Stickman} mozgását.
    \item[Kontextusváltás] A \emph{Játékos} utasítására bármikor kontextusváltás történhet, mely bekövetkeztekor a grafikus felület ráközelít az \emph{Aktuális keret}re (\emph{Pálya kontextus}ról \emph{Keret kontextus}ra váltás esetén) vagy eltávolodik az a \emph{Aktuális keret}ről az összes keret nézetére (\emph{Pálya kontextus})
    \item[Következő pálya] Az aktuális \emph{Pálya} után következő \emph{Pálya} a \emph{Pályalistá}ban. Amennyiben nincs következő pálya a listában, a következő pályának a készítők nevét listázó képernyő tekinthető.
    \item[Pályalista] Az program által tartalmazott összes \emph{Pálya} egy előre definiált sorrendben.

\end{description}

\subsection{Essential use-case-ek}
    %insert figure
	%\begin{figure}[h]
	%	\begin{center}
	%		\includegraphics[scale=0.5]{----FILENAME-HERE----}
	%		\caption{UML diagram test}
	%	\end{center}
	%\end{figure}

\end{document}
