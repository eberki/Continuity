\documentclass[11pt,a4paper,oneside]{article}
\usepackage[utf8]{inputenc}
\usepackage{t1enc} % hyphenate accented chars
\usepackage[hungarian]{babel}
\usepackage{../fedlap}
\usepackage{fancyhdr} % elofej, elolab
\usepackage{graphicx}
\setcounter{secnumdepth}{3} % enable subsubsection

% hasonlitson a doc verziora
\addtolength{\voffset}{-1cm}

% cim
\csapat{nand}{404}
\konzulens{Bozóki Szilárd}
\datum{\today}

% csapattagok
\taga{Berki Endre}{HQNHER}{berkiendre@gmail.com}
\tagb{Fodor Bertalan Ferenc}{H4T1UX}{foberci@gmail.com}
\tagc{Kádár András}{JFENWR}{arycika@gmail.com}
\tagd{Thaler Benedek}{EDDO10}{thalerbenedek@gmail.com}

\setlength{\headheight}{1.3em}
\setlength{\headsep}{2em}

% elofej, elolab
\fancyhf{}

\fancyhead[OL] { \tiny \leftmark{} }
\fancyhead[OR] { \tmpcsapat }

\fancyfoot[OR] { \thepage }
\fancyfoot[OL] { \tmpdatum }

\pagestyle{fancy}


\begin{document}

\anyag{2. Követelmény, projekt, funkcionalitás}
\fedlap
\tableofcontents

\addtocounter{section}{1}
\section{Követelmény, projekt, funkcionalitás}

\subsection{Követelmény definíció}

    \subsubsection{A program bemutatása, alapvető feladatai}
Az elkészített program egy nagymértékben logikai, kismértékben ügyességi játék, mely koncepcióját tekintve teljesen megfelel a Ragtime Games által fejlesztett Continuity játéknak\footnote{http://continuitygame.com/playcontinuity.html}. A játék során a játékos több pályán keresztül tesztelheti logikai és problémamegoldó készségét. Az előre kialakított pályákon az előrehaladás a keretek egymáshoz képest történő elmozdításával, a kereteken belül pedig egy figurával 2 dimenzióban végzett mozgással történik. Egy pálya akkor tekinthető teljesítettnek, ha a felhasználó összegyűjtötte a pályán található összes kulcsot és ezek felhasználásával bemegy a pálya végét szimbolizáló ajtón.

Minden pálya egy n*n-es táblából áll, amin n*n-1 keret található. Az összeillő keretek között átjárás van, az üres hellyel szomszédos keretek átmozgathatóak az üres helyre. A játékos által irányított figurának a keretekben található kulcsokat kell összeszednie, hogy azzal a szintén a keretekben található ajtót kinyitva a pályát teljesítse.

    \subsubsection{Felhasználói felület}
A kész program grafikus felhasználói felülettel rendelkezik, melyen az indítással és működéssel kapcsolatos funkciók egy menüsoron keresztül érhetőek el, egér vagy billentyűzet segítségével. Maga a játék kizárólag billentyűzettel irányítható.

    \subsubsection{A program futtatásának követelményei}
A program futtatásához szükséges a futtató számítógépre telepített Java Runtime Environment (JRE), mely ingyenesen beszerezhető a gyártó honlapjáról\footnote{http://java.com/en/download/index.jsp} a legtöbb platformra. A program hardware követelménye megegyezik az Oracle által meghatározott minimum-konfigurációval\footnote{http://www.java.com/en/download/help/sysreq.xml}, mely szükségessé tesz egy kompatibilis operációs rendszert (Windows, Linux), rendszertől függően 64-128 MB memóriát és hozzávetőlegesen 100 MB szabad helyet.

    \subsubsection{A program telepítése}
A program futtatásához a kész terméket csupán a futtató számítógép merevlemezére kell másolni, további telepítés vagy beállítás nem szükséges, amennyiben a futtató számítógép megfelel a futtatás követelményeinek.

    \subsubsection{A program fordítása}
A program fordításához szükség van a \texttt{javac} programra, mellyel a \texttt{*.java} fájlok lefordíthatóak, és a futtatható \texttt{jar} fájl elkészíthető.
    
\subsection{Projekt terv}
\subsection{Feladatleírás}
\subsection{Szótár}
\subsection{Essential use-case-ek}
    %insert figure
	%\begin{figure}[h]
	%	\begin{center}
	%		\includegraphics[scale=0.5]{----FILENAME-HERE----}
	%		\caption{UML diagram test}
	%	\end{center}
	%\end{figure}

\end{document}
